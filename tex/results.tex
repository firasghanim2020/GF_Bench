\section{Results}

The technique described in this paper was implemented in ABC \cite{abc-link}. It applies algebraic rewriting to the AIG and generates the polynomial signature. The experiments were conducted on a PC with Intel(R) Xeon CPU E5-2420 v2 2.20 GHz x12 with 32 GB memory. The experiments include gate-level rewriting of Carry-Save-Adder (CSA) multipliers and {\color{red}radix-4 Booth multipliers}, up to 512 bits. The results are compared with \textit{functional extraction} \cite{ciesielski2015verification} and {\color{red}the Grobner Basis based approach \cite{sayedformal:date-2016}\cite{pruss2015TCAD:efficient}, using post-synthesized circuits.} 
The results show that the proposed technique is more efficient than the state-of-the-art technique for extracting the polynomial expressions for those arithmetic circuits. 

We evaluate our AIG-based algebraic rewriting approach using pre-synthesized and post-synthesized CSA multipliers shown in Table \ref{tbl:pre-post-tbl}, and post-synthesized complex arithmetic circuits shown in Table \ref{tbl:complex-circuits}. The gate-level arithmetic circuits are unsigned and are taken from \cite{ciesielski2015verification}. Booth multipliers are generated by \textit{\%blast -b} command using ABC \cite{abc-link}. The runtime and memory usage are compared to \textit{functional extraction} \cite{ciesielski2015verification}. In Table \ref{tbl:complex-circuits}, the functions of the arithmetic circuits are shown in the first column.
The bit-width varies between 64 and 512 bits\footnote{512-bit post-synthesized multipliers are not reported in \cite{ciesielski2015verification}.}. We can see that the runtime of the proposed approach outperforms other approaches for the post-synthesized multipliers for any bit-width. The memory usage has been reduced on average 60\%, compared to \textit{function extraction} \cite{ciesielski2015verification}. Also, the complexity of extracting the polynomial expression using functional extraction is increased when the multipliers are synthesized. For example, extracting post-synthesized 256-bit multiplier using functional extraction requires 9x more runtime and more memory. However, using the proposed approach, the runtime of extracting pre- or post-synthesized multipliers are almost the same. More importantly, we can see that our approach surpass \textit{functional extraction} on complex arithmetic designs (Table \ref{tbl:complex-circuits}).

\begin{table}[t]
\color{red}
\scriptsize
\centering
\caption{Results of applying AIG-based algebraic rewriting to pre- and post-synthesized CSA multipliers compared to \textit{functional extraction} presented in \cite{ciesielski2015verification}. \textit{*t(s)} is the runtime in seconds. \textit{*mem} is the memory usage in mb. {\it *256B} is 256-bit Booth multiplier. {\it TO} = out of 24 hours.}
\vspace{-3mm}
\label{tbl:pre-post-tbl}
\begin{tabular}{|r|c|r|r|r|c|c|c|r|r|}
\hline
\multicolumn{1}{|l|}{\multirow{3}{*}{\# bits}} & \multicolumn{4}{c|}{\textit{Pre-Syntheized}} & \multicolumn{5}{c|}{\textit{Post-Synthesized}} \\ \cline{2-10} 
\multicolumn{1}{|l|}{} & \multicolumn{2}{c|}{\cite{ciesielski2015verification}} & \multicolumn{2}{l|}{This approach} & \multirow{2}{*}{\begin{tabular}[c]{@{}c@{}}\cite{ciesielski2015verification}\\ t(s)\end{tabular}} & \multirow{2}{*}{\begin{tabular}[c]{@{}c@{}}\cite{sayedformal:date-2016}\\ t(s)\end{tabular}} & \multirow{2}{*}{\begin{tabular}[c]{@{}c@{}}\cite{pruss2015TCAD:efficient}\\ t(s)\end{tabular}} & \multicolumn{2}{c|}{This approach} \\ \cline{2-5} \cline{9-10} 
\multicolumn{1}{|l|}{} & t(s) & \multicolumn{1}{c|}{mem} & \multicolumn{1}{c|}{t(s)} & \multicolumn{1}{c|}{mem} &  &  &  & \multicolumn{1}{c|}{t(s)} & \multicolumn{1}{c|}{mem} \\ \hline
64 & \multicolumn{1}{r|}{1.9} & 74 & 0.08 & 34 & \multicolumn{1}{r|}{5.50} & 593 & TO & 0.11 & 34 \\ \hline
128 & \multicolumn{1}{r|}{8.1} & 288 & 0.78 & 117 & \multicolumn{1}{r|}{39.6} & TO & TO & 0.91 & 120 \\ \hline
256 & \multicolumn{1}{r|}{32.6} & 1157 & 7.80 & 441 & \multicolumn{1}{r|}{285} & TO & TO & 8.23 & 439 \\ \hline
256B & TO & \multicolumn{1}{c|}{-} & 33.7 & 423 & TO & TO & TO & 39.5 & 431 \\ \hline
512 & \multicolumn{1}{r|}{130} & 4427 & 31.7 & 1695 & - & - & - & \multicolumn{1}{c|}{-} & \multicolumn{1}{c|}{-} \\ \hline
\end{tabular}
\end{table}
\begin{table}[]
\centering
\scriptsize
\caption{Results of applying AIG-based algebraic rewriting to post-synthesized complex arithmetic circuits compared to \textit{functional extraction} presented in \cite{ciesielski2015verification}. \textit{*MO} = Memory out of 8 GB.}
\vspace{-3mm}
\label{tbl:complex-circuits}
\begin{tabular}{|l|r|r|r|r|}
\hline
\multicolumn{1}{|c|}{\multirow{2}{*}{\begin{tabular}[c]{@{}c@{}}Benchmarks\\ (256-bit)\end{tabular}}} & \multicolumn{2}{c|}{{[}1{]}} & \multicolumn{2}{c|}{This approach} \\ \cline{2-5} 
\multicolumn{1}{|c|}{} & \multicolumn{1}{c|}{runtime(s)} & \multicolumn{1}{c|}{mem(MB)} & \multicolumn{1}{c|}{runtime(s)} & \multicolumn{1}{c|}{mem(MB)} \\ \hline
\textit{F=A$\times$B+C} & 179.1 & 1182 & 5.1 & 447 \\ \hline
\textit{F=A$\times$(B+C)} & 209.3 & 1120 & 5.1 & 451 \\ \hline
\textit{F=A$\times$B$\times$C} & - & MO & 37.5 & 2871 \\ \hline
\textit{F=1+A+$A^2$+$A^3$} & - & MO & 47.1 & 3331 \\ \hline
\end{tabular}
\end{table}


%% Please add the following required packages to your document preamble:
% \usepackage{multirow}
\begin{table}[t]
\centering
\caption{Results of applying AIG-based algebraic rewriting on pre-synthesized CSA-multipliers compared to \textit{function extraction} \cite{ciesielski2015verification}.}
\vspace{-1mm}
\label{tbl:pre}
\begin{tabular}{|r|r|r|r|r|}
\hline
\multirow{2}{*}{\# bits} & \multicolumn{2}{c|}{{[}1{]}} & \multicolumn{2}{c|}{This approach} \\ \cline{2-5} 
 & runtime(s) & mem(MB) & runtime(s) & mem(MB) \\ \hline
8 & 0.02 & 4.9 & 0.01 & 9.8 \\ \hline
16 & 0.10 & 8.2 & 0.01 & 10.4 \\ \hline
64 & 1.89 & 73.9 & 0.04 & 34.2 \\ \hline
128 & 8.12 & 288.4 & 0.15 & 117.2 \\ \hline
256 & 32.65 & 1157.3 & 0.82 & 440.5 \\ \hline
512 & 130.22 & 4427.5 & 3.76 & 1695.1 \\ \hline
\end{tabular}
\end{table}

%\begin{table}[!htb]
\centering
\caption{Results of applying AIG-based algebraic rewriting on post-synthesized CSA-multipliers compared to \textit{function extraction} \cite{ciesielski2015verification}.}
\label{tbl:post}
\begin{tabular}{|r|r|r|r|r|}
\hline
\multirow{2}{*}{\# bits} & \multicolumn{2}{c|}{{[}1{]}} & \multicolumn{2}{c|}{This approach} \\ \cline{2-5} 
 & runtime(s) & mem(MB) & runtime(s) & mem(MB) \\ \hline
8 & 0.04 & 2.9 & 0.01 & 9.7 \\ \hline
16 & 0.14 & 6.1 & 0.01 & 10.4 \\ \hline
64 & 5.50 & 76.3 & 0.04 & 34.3 \\ \hline
128 & 39.64 & 299.2 & 0.16 & 120.0 \\ \hline
256 & 285.22 & 1250.6 & 0.82 & 438.9 \\ \hline
\end{tabular}
\end{table}

%% Please add the following required packages to your document preamble:
% \usepackage{multirow}
\begin{table}[]
\centering
\caption{My caption}
\label{my-label}
\begin{tabular}{|r|r|r|r|r|}
\hline
\multirow{2}{*}{n-bit} & \multicolumn{2}{r|}{{[}1{]}} & \multicolumn{2}{r|}{This approach} \\ \cline{2-5} 
 & runtime(s) & mem(MB) & runtime(s) & mem(MB) \\ \hline
16 &  &  &  &  \\ \hline
32 &  &  &  &  \\ \hline
64 &  &  &  &  \\ \hline
128 &  &  &  &  \\ \hline
256 &  &  &  &  \\ \hline
512 &  &  &  &  \\ \hline
\end{tabular}
\end{table}

