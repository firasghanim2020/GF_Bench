\section{Results}

The technique described in this paper was implemented in C, and embedded in the ABC system\cite{abc-link}. It applies algebraic rewriting on AIG network and generates the polynomial signature. The experimental results include rewriting on gate-level Carry-save-adder (CSA) multipliers up to 512-bit, and compared to \textit{function extraction} \cite{ciesielski2015verification}.
They show that our technique is more efficient than the state-of-the-art technique in extracting the polynomial expressions of CSA multipliers. The experiments were conducted on a PC with Intel(R) Xeon CPU E5-2420 v2 2.20 GHz x12 with 32 GB memory.

The results of applying AIG-based algebraic rewriting on pre-synthesized and post-synthesized CSA-multipliers are shown in Table \ref{tbl:pre} and Table \ref{tbl:post}. The gate-level multipliers are taken from \cite{ciesielski2015verification}. Runtime and memory usage of \textit{function extraction} \cite{ciesielski2015verification} are shown in columns 2 and 3, and the results of the described approach are shown in columns 4 and 5. The bit-width varies from 8 to 256 bits\footnote{512-bit post-synthesized multipliers was not reported in \cite{ciesielski2015verification}.}. First, we can see that the runtime of the proposing approach is lower than a second on both the pre- and post-synthesized CSA-multipliers from 8-bit to 256-bit. At the same time, the memory usage has been reduced on average 60\% compared to \textit{function extraction} \cite{ciesielski2015verification}. Finally, the complexity of extracting the polynomial expression using function extraction is increased while the multipliers have been synthesized. For example, extracting post-synthesized 256-bit multiplier using function extraction requires 9x more runtime and more memory. However, using the proposed approach, the runtime of extracting pre- or post-synthesized multipliers are almost the same.

% Please add the following required packages to your document preamble:
% \usepackage{multirow}
\begin{table}[t]
\centering
\caption{Results of applying AIG-based algebraic rewriting on pre-synthesized CSA-multipliers compared to \textit{function extraction} \cite{ciesielski2015verification}.}
\vspace{-1mm}
\label{tbl:pre}
\begin{tabular}{|r|r|r|r|r|}
\hline
\multirow{2}{*}{\# bits} & \multicolumn{2}{c|}{{[}1{]}} & \multicolumn{2}{c|}{This approach} \\ \cline{2-5} 
 & runtime(s) & mem(MB) & runtime(s) & mem(MB) \\ \hline
8 & 0.02 & 4.9 & 0.01 & 9.8 \\ \hline
16 & 0.10 & 8.2 & 0.01 & 10.4 \\ \hline
64 & 1.89 & 73.9 & 0.04 & 34.2 \\ \hline
128 & 8.12 & 288.4 & 0.15 & 117.2 \\ \hline
256 & 32.65 & 1157.3 & 0.82 & 440.5 \\ \hline
512 & 130.22 & 4427.5 & 3.76 & 1695.1 \\ \hline
\end{tabular}
\end{table}

\begin{table}[!htb]
\centering
\caption{Results of applying AIG-based algebraic rewriting to post-synthesized CSA-multipliers compared to \textit{functional extraction} \cite{ciesielski2015verification}.}
\label{tbl:post}
\begin{tabular}{|r|r|r|r|r|}
\hline
\multirow{2}{*}{\# bits} & \multicolumn{2}{c|}{{[}1{]}} & \multicolumn{2}{c|}{This approach} \\ \cline{2-5} 
 & runtime(s) & mem(MB) & runtime(s) & mem(MB) \\ \hline
8 & 0.04 & 2.9 & 0.01 & 9.7 \\ \hline
16 & 0.14 & 6.1 & 0.01 & 10.4 \\ \hline
64 & 5.50 & 76.3 & 0.04 & 34.3 \\ \hline
128 & 39.64 & 299.2 & 0.16 & 120.0 \\ \hline
256 & 285.22 & 1250.6 & 0.82 & 438.9 \\ \hline
\end{tabular}
\end{table}

%% Please add the following required packages to your document preamble:
% \usepackage{multirow}
\begin{table}[]
\centering
\caption{My caption}
\label{my-label}
\begin{tabular}{|r|r|r|r|r|}
\hline
\multirow{2}{*}{n-bit} & \multicolumn{2}{r|}{{[}1{]}} & \multicolumn{2}{r|}{This approach} \\ \cline{2-5} 
 & runtime(s) & mem(MB) & runtime(s) & mem(MB) \\ \hline
16 &  &  &  &  \\ \hline
32 &  &  &  &  \\ \hline
64 &  &  &  &  \\ \hline
128 &  &  &  &  \\ \hline
256 &  &  &  &  \\ \hline
512 &  &  &  &  \\ \hline
\end{tabular}
\end{table}
